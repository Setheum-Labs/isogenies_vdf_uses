\documentclass{llncs}

\usepackage{amsmath,amsfonts,amssymb}
\usepackage{hyperref}
\hypersetup{
	unicode=true,
	colorlinks=true,
	citecolor=blue!70!black,
	filecolor=black,
	linkcolor=red!70!black,
	urlcolor=blue,
	pdfstartview={FitH},
}

\newcommand{\Z}{\mathbb{Z}}
\newcommand{\Q}{\mathbb{Q}}
\newcommand{\F}{\mathbb{F}}
\DeclareMathOperator{\poly}{poly}
\DeclareMathOperator{\polylog}{polylog}
\DeclareMathOperator{\Setup}{\mathsf{Setup}}
\DeclareMathOperator{\Extract}{\mathsf{Extract}}
\DeclareMathOperator{\Encaps}{\mathsf{Encaps}}
\DeclareMathOperator{\Decaps}{\mathsf{Decaps}}
\newcommand{\pp}{\mathsf{pp}}
\newcommand{\id}{\mathsf{id}}
\newcommand{\idk}{\mathsf{idk}}
\newcommand{\keyspace}{\mathcal{K}}
\newcommand{\cipherspace}{\mathcal{C}}

\title{Delay Encryption}
\author{Jeffrey Burdges\inst{1}
  \and
  Luca De Feo\inst{2}\orcidID{0000-0002-9321-0773}}
\institute{
  Web 3, Switzerland
  \and
  IBM Research Zürich, Switzerland
}

\begin{document}

\maketitle

\begin{abstract}
  We introduce a new primitive named Delay Encryption, and give an
  efficient instantation based on isogenies of supersingular curves
  and pairings. %
  Delay Encryption is related to Time-lock Puzzles and Verifiable
  Delay Functions, and can be roughly described as ``identity based
  encryption with slow decryption''. %
  It has several applications in distributed protocols, such as
  sealed bid Vickrey auctions and...

  We give an instantiation of Delay Encryption by modifying Boneh and
  Frankiln's IBE scheme, where we replace one secret by a long chain
  of isognies, as in the isogeny VDF of De Feo, Masson, Petit and
  Sanso. %
  Similarly to the isogeny-based VDF, our Delay Encryption requires a
  trusted setup before parameters can be safely used; our trusted
  setup is identical to that of the VDF, thus the same parameters can
  be generated once and shared for many executions of both protocols,
  with possibly different delay parameters.

  We also discuss several topics around delay protocols
  based on isogenies that were left untreated by De Feo \emph{et al.},
  namely: distributed trusted setup, watermarking, and implementations
  issues.
\end{abstract}

\section{Introduction}
\label{sec:introduction}

% Recall that an Identity Based Encryption scheme (IBE) is a public
% key encryption

\subsection{Applications of Delay Encryption}


\section{Definitions}
\label{sec:definitions}

Our definition of Delay Encryption uses an API similar to a Key
Encapsulation Mechanism; the adaptation to a PKE-like API is
straightforward. A Delay Encryption scheme consists of four
algorithms: $\Setup$, $\Extract$, $\Encaps$ and $\Decaps$:

\begin{description}
\item[$\Setup(\lambda, T) \to \pp$.] %
  Takes a \emph{security parameter} $\lambda$, a \emph{delay
    parameter} $T$, and produces a set of public parameters $\pp$. %
  $\Setup$ must run in time $\poly(\lambda,T)$.
\item[$\Extract(\pp,\id) \to \idk$.] %
  Takes the public parameters $\pp$ and a \emph{session identifier}
  $\id\in\{0,1\}^*$, and outputs a \emph{session key} $\idk$. %
  $\Extract$ is expected to run in time \emph{exactly} $T$, see below.
\item[$\Encaps(\pp,\id)\to (c,k)$.] %
  Takes the public parameters $\pp$ and a \emph{session identifier}
  $\id\in\{0,1\}^*$, and outputs a \emph{cyphertext}
  $c\in\cipherspace$ and a \emph{key} $k\in\keyspace$. %
  $\Encaps$ must run in time $\poly(\lambda)$.
\item[$\Decaps(\pp,\id,\idk,c)\to k$.] %
  Takes the public parameters $\pp$, a \emph{session identifier}
  $\id$, a \emph{session key} $\idk$, a ciphertext $c\in\cipherspace$,
  and outputs key $k\in\keyspace$. %
  $\Decaps$ must run in time $\poly(\lambda)$.
\end{description}

A Delay Encryption scheme is correct if
\[\bigl(c,\Decaps(\pp,\id,\idk,c)\bigr) = \Encaps(\pp,\id),\]
whenever $\idk=\Extract(\pp,\id)$. %
The security of Delay Encryption is defined similarly to that of
public key encryption schemes, and in particular of identity-based
ones; however one additional property is required of $\Extract$: that
for a randomly selected identifier $\id$, the probability that
$\Extract$ outputs $\idk$ in time less than $T$ is negligible. %
We now give the formal definition.

\paragraph{The security game.} It is apparent from the definitions
that Delay Encryption has no secrets: after public parameters $\pp$
are generated, anyone can run any of the algorithms. %
Thus, the usual notion of indistinguishability will only be defined
with respect to the delay parameter $T$: no adversary is able to
distinguish a key $k$ from a random string in time $T-o(t)$, but
anyone can in time $T$. %
Properly defining what is meant by ``time'' requires fixing a
computation model. %
Here we follow the usual convention from VDFs, and assume a model of
parallel computation: in this context, ``time $T$'' may mean $T$ steps
of a parallel Turing machine, or an arithmetic circuit of depth $T$. %
Crucially, we do not bound the amount of parallelism of the Turing
machine, or the breadth of the circuit, i.e., we focus on
\emph{sequential delay} functions.

We consider the following $\Delta$-IND-CCA game. %
Note that the game involves no oracles, owing to the fact that the
scheme has no secrets.

\begin{description}
\item[Precomputation.] The adversary receives $\pp$ as input, and
  outputs an algorithm $\mathcal{A}_1$. %
\item[Challenge.] The challenger selects a random $\id$ and computes
  $(c,k_0)\gets\Encaps(\pp,\id)$. %
  It then picks a uniformly random $k_1\in\keyspace$, and a random bit
  $b\in\{0,1\}$. %
  Finally, it outputs $(c,k_b)$.
\item[Guess.] The algorithm $\mathcal{A}_1$ is run on input
  $(c,k_b)$. %
  The adversary wins if $\mathcal{A}_1$ terminates in time less than
  $\Delta$, and the output is such that $\mathcal{A}_1(c,k_b) = b$.
\end{description}

We say a Delay Encryption scheme is \emph{$\Delta$-Delay
  Indistinguishable under Chosen Ciphertext Attacks} if, for any
efficient adversary running the precomputation in time
$\poly(\lambda,T)$, the probability of winning the game is
negligible. %
Obviously, the interesting schemes are those where $\Delta = t-o(t)$.

\subsection{Instantiation from supersingular isogenies and pairings}



\section{Distributed trusted setup}
\label{sec:distr-trust-setup}

\section{Watermarking}
\label{sec:watermarking}

\section{Challenges in implementing isogeny-based delay functions}
\label{sec:secure-impl-isog}

\section{Conclusion}

\bibliography{isovdf,isogenies_bib/isogenies}
\bibliographystyle{splncs04}

\end{document}

% LocalWords:  bilinear instantiation VDF subexponential morphisms
% LocalWords:  instantiations supersingular endomorphism morphism
% LocalWords:  isogenous homomomorphism endomorphisms homomorphism
% LocalWords:  isogenies Frobenius isogeny subgraphs distorsion
% LocalWords:  prover soundess sequentiality quaternion

